% -*- compile-command: "make HOCKING-RGSOC.pdf" -*-
\documentclass{beamer}
\usepackage{tikz}
\usepackage[all]{xy}
\usepackage{amsmath,amssymb}
\usepackage{hyperref}
\usepackage{graphicx}
\usepackage{algorithmic}

\DeclareMathOperator*{\argmin}{arg\,min}
\DeclareMathOperator*{\Lik}{Lik}
\DeclareMathOperator*{\PoissonLoss}{PoissonLoss}
\DeclareMathOperator*{\Peaks}{Peaks}
\DeclareMathOperator*{\Segments}{Segments}
\DeclareMathOperator*{\argmax}{arg\,max}
\DeclareMathOperator*{\maximize}{maximize}
\DeclareMathOperator*{\minimize}{minimize}
\newcommand{\sign}{\operatorname{sign}}
\newcommand{\RR}{\mathbb R}
\newcommand{\ZZ}{\mathbb Z}
\newcommand{\NN}{\mathbb N}

% Set transparency of non-highlighted sections in the table of
% contents slide.
\setbeamertemplate{section in toc shaded}[default][100]
\AtBeginSection[]
{
  \setbeamercolor{section in toc}{fg=red} 
  \setbeamercolor{section in toc shaded}{fg=black} 
  \begin{frame}
    \tableofcontents[currentsection]
  \end{frame}
}

\begin{document}

\title{R project in Google Summer of code (R-GSOC)}

\author{
  Toby Dylan Hocking\\
  toby.hocking@r-project.org\\
\url{https://github.com/rstats-gsoc/gsoc2017/wiki}}

\date{26 May 2017}

\maketitle

\begin{frame}
  \frametitle{Google Summer of Code (GSOC)}
  Student gets paid for writing free/open-source code for 3 months.
  \begin{description}
  \item[Jan] \textbf{Admins} for organizations e.g. R,
    LibreOffice, MariaDB, OpenStreetMap etc apply to Google.\\
    \url{https://summerofcode.withgoogle.com/organizations/}
  \item[Feb] \textbf{Mentors} propose projects for each org.
  \item[Mar]\textbf{Students} submit project proposals to Google.
  \item[Apr] Google gives funding for $n$ students to an org.
  \item[May] The top $n$ students are selected.
  \item[June] Students begin coding.
  \item[July] Midterm evaluation.
  \item[Aug] Final evaluation.
  \end{description}

  I have participated since 2012 as an \textbf{admin} and
  \textbf{mentor} for the R project.
\end{frame}

%\section{Participate as a mentor}

\begin{frame}
  \frametitle{Mentors propose 3-month coding projects}
  Coding projects should:
  \begin{itemize}
  \item Result in free/open-source software.
  \item Be 3 months of full time work for a student.
  \item Include writing documentation and tests.
  \item Not include original research.
  \end{itemize}
  R GSOC projects typically involve package development: 
  \begin{itemize}
  \item Porting/interfacing code from some other language, e.g. re2r
    package interfaces RE2 C++ library, Qin Wenfeng 2016.
  \item Coding an algorithm from a research paper, e.g. bigoptim
    package for Stochastic Average Gradient algorithm, Ishmael
    Belghazi 2015.
  \item Extending some existing package, e.g. postCP package, Malith
    Jayaweera 2016.
  \end{itemize}
\end{frame}

\begin{frame}
  \frametitle{Mentors write coding project proposals}
  Each proposal should be posted to our wiki with:
  \begin{description}
  \item[Background/related work] What problem do you want to solve,
    and why aren't existing R packages good enough?
  \item[Coding plan] Detailed outline of functionality to implement.
  \item[Expected impact] Will the package be useful for the R
    community?
  \item[Mentor bios] contact info and qualifications (package dev and
    GSOC experience). Two mentors required! (ideally from different institutions)
  \item[Tests] exercises that students can do to prove they are
    capable of the coding project. The harder the tests, the easier it
    will be to choose the right student!
  \end{description}

\url{https://github.com/rstats-gsoc/gsoc2017/wiki/table-of-proposed-coding-projects}
\end{frame}

\begin{frame}
  \frametitle{Mentor obligations}
  \begin{description}
  \item[Jan] Read our wiki/FAQs and sign up to the gsoc-r list.\\
    {\small \url{https://github.com/rstats-gsoc/gsoc2017/wiki}}
  \item[Feb] Write project proposal on our wiki.
  \item[Mar] Login to google system to comment on submitted applications.
  \item[June-Aug] Mentoring students via weekly skype calls.
  \end{description}
  Throughout, you must be available and responsive to student
  questions via email! 
  \begin{itemize}
  \item The primary goal of GSOC is to teach the students about
    free/open-source R package development.
  \item Advancing your particular project should be considered a
    secondary concern.
  \end{itemize}
\end{frame}

%\section{Participate as a student}

\begin{frame}
  \frametitle{Student timeline}
  \begin{description}
  \item[Jan] Check the wiki for project proposals, start working on project tests.
  \item[Feb] Submit test solutions to mentors, and tell them why you
    are interested in the project.
  \item[Mar] Write an application using our template, and submit it to Google.
  \item[May] If you are selected, begin ``bonding'' with mentors and
    preparing for the coding project.
  \item[Jun] Coding period starts. Commit and push daily. Ask mentors
    questions via email and weekly skype calls. Expect to work about
    40 hours per week.
  \item[July-Aug] Midterm and final evaluations -- should be no
    problem if you are in touch with mentors.
  \end{description}

  Students get paid an amount which depends on the country of their
  university.
\end{frame}

%\section{Some analysis}

\begin{frame}[fragile]
  \frametitle{Students who came back as mentors this year}
  For example, 
  \begin{itemize}
  \item Akash Tandon wrote Rperform as a student in 2015-2016, now
    lead mentor on same project.
  \item Carson Sievert improved Animint as a student in 2014,
    now co-mentoring 2015-2017.
  \item Vijay Barve worked on biodiversity data packages as a student
    in 2012-2015, now a co-mentor 2016-2017.
  \item Narayani Barve worked on Ecological Niche Models as a student
    in 2014, now a co-mentor 2016-2017.
  \item Qiang Kou extended mzR as a student in 2014, now co-mentoring
    machine learning projects 2016-2017.
  \end{itemize}
% \begin{verbatim}
%     student.project     years               person year project.mentored
%  1: A GUI for Graph      2010          Ian Fellows 2011  Image Analysis 
%  2: CAMEL: Calibrat 2011-2013             tourzhao 2014  GenPCA: A Gener
%  3: CAMEL: Calibrat 2011-2013             tourzhao 2015             SDCA
%  4: CAMEL: Calibrat 2011-2013             tourzhao 2016  Biconvex minimi
%  5: HyperSpec: Para      2012         Simon Fuller 2013  Implement/Port 
%  6: HyperSpec: Para      2012         Simon Fuller 2016  Hyperspectral U
%  7: Improvements to      2012     Michael Weylandt 2013  Improvements to
%  8: xtend RTAQ for       2012 Jonathan Cornelissen 2013  Highfrequency: 
%  9: xtend RTAQ for       2012 Jonathan Cornelissen 2014  Spot volatility
% 10: bdvis: Biodiver 2012-2015          Vijay Barve 2016  Boundary detect
% 11: bdvis: Biodiver 2012-2015          Vijay Barve 2016     NicheToolbox
% 12: Improve renderi      2013     Susan VanderPlas 2014          animint
% 13: Improve renderi      2013     Susan VanderPlas 2015          Animint
% 14:   Extending mzR      2014            Qiang Kou 2016  Deep learning w
% 15: Tools for pre a      2014       Narayani Barve 2016     NicheToolbox
% 16:         animint      2014       Carson Sievert 2015          Animint
% 17:         animint      2014       Carson Sievert 2016          Animint
% \end{verbatim}
\end{frame}


\begin{frame}
  \frametitle{Projects over time}
  \includegraphics[width=\textwidth]{figure-projects}
\end{frame}

\end{document}
